\subsection{How did it appear?}

Dart is an open source script language focused in web applications and developed
by Google\cite{3_1}. It was announced at GOTO
Conference 2011 (Aarhus, Denmark)\cite{3_2} and
its first stable release, Dart 1.0, comes in November
2013\cite{3_3}.

It was designed by Lars Bak and Kasper Lund, both of them employed by Google.
Lars has contributed to the Chrome Browser project developing the V8 JavaScript
Engine\cite{3_4}, the open source JavaScript
engine that runs Google's browser and gave birth to some other big projects like
Node.js\cite{3_5}.

### Why was it designed and implemented?

In the words of Google\cite{3_6}:

> At Google we’ve written our share of web apps, and we’ve tried in many ways to
> make improvements to that development process, short of introducing a new
> language. Now we think it’s time to take that leap. We designed Dart to be
> easy to write development tools for, well-suited to modern app development,
> and capable of high-performance implementations.

But in pragmatic terms all this means that Google thinks that JavaScript is
unproductive and his proposal to improve it was to design a better language for
the browsers. Google is investing a lot of effort on both sides, JavaScript
and Dart\cite{3_7} and believes that developers
should have a choice when they build for the web. Adding a new option, such as
Dart, does not imply replacing the JavaScript already existing 
option\cite{3_8}.

Dart has its own Virtual Machine which enable the Dart programs to run on every
systems. There is also a **dar2js** compiler which makes Dart code compile to
JavaScript code and run on modern browsers. And there is the third option, the
Dartium\cite{3_9} project that brings the Dart VM
inside a Chromium Browser, so Dart can be used directly as a browser's script
language like JavaScript does at Chrome nowadays.

The Dart team isn't focused in bring Dart to Google Chrome but in compiling Dart
to JavaScript. They have decided not to integrate the Dart VM into
Chrome but to generate a better JavaScript compilation\cite{3_10}
\cite{3_11}.

### How can we use it, e.g., install and run?

Dart has it's own Virtual Machine and has easy installations on all major
systems. The current stable version of Dart is **1.12.2**. For installation on a
**Mac** you can use Homebrew:

```
brew tap dart-lang/dart
brew install dart
```

Also you can install the Dartium browser with:

```
brew tap dart-lang/dart
brew install dart --with-content-shell --with-dartium
```

For **Linux** there are **Debian** packages and these are the commands for
installation on the Debian based systems:

```
sudo apt-get install apt-transport-https
sudo sh -c 'curl https://dl-ssl.google.com/linux/linux_signing_key.pub | apt-key add -'
sudo sh -c 'curl https://storage.googleapis.com/download.dartlang.org/linux/debian/dart_stable.list > /etc/apt/sources.list.d/dart_stable.list'
sudo apt-get update
sudo apt-get install dart
```

There are other options to build and install on **various Linux** distros\cite{3_12}.

There is also the versions of the SDK where you can download the portable
package and link to your system's path whatever you want:
[https://www.dartlang.org/downloads/archive/](https://www.dartlang.org/downloads/archive/).

Now let's check if the installation works. In any path (Dart doesn't force
directory structure) create the file called _hello.dart_ with the following
content:

```dart
main() {
  print("Hello Dart!");
}
```

Run with ```dart hello.dart``` and the "Hello Dart!" message must show up. Now
test the JavaScript compilation by running, with the same _hello.dart_ file, the
command:

```
dart2js hello.dart
```

You might have the message _Dart file (hello.dart) compiled to JavaScript:
out.js_. The contents of the file _out.js_ is huge if compared to the initial
Dart code. We will have more details of JavaScript compilation in the future.
Now,if you have a JavaScript runtime in your system, like Node.js, you will have
the same result of running the _hello.dart_ file if you run:

```
node out.js
```

If you don't have a JavaScript runtime in your system, you'll test the
JavaScript compilation by creating a _test.html_ file in the same directory
of the _out.js_ file with the following content:

```html
<script type="text/javascript" src="out.js"></script>
```

Then open the file in a modern browser (like Google Chrome or Mozilla Firefox)
and open the browser console (by pressing the F12 key) and the message _Hello
Dart!_ is printed in the browser's console.

### Simple programs

I will demonstrate some features of the language by writing some simple
programs and analysing some points right after it.

#### Dart Packages

Every Dart program is a library, even if it the program is not written using
libraries definitions. Libraries not only provide APIs, but are a unit of
privacy: identifiers that start with an underscore (\_) are visible only
inside the library (like private or protected members in Java).

Libraries can be distributed using packages. Dart has a powerful package
manager that resolve package dependencies and package installation as we
specify our project dependencies. Dart has also a huge repository of packages[https://pub.dartlang.org/]
that delivers the packages listed as project's dependencies.

Let's implement a Dart example program that uses the package manager to install
a library. This is the code of the program _request.dart_:

```dart
import 'dart:math';
import 'package:http/http.dart' as http;

readFromRemoteFile(url, sentence) {
  var rng = new Random();
  print(sentence);
  http.get(url).then((val){
    var texto = val.body;
    var frases = texto.split("\n");
    print(frases[rng.nextInt(frases.length)]);
  });
}

ramones() {
  readFromRemoteFile('http://aurelio.net/doc/ramones.txt',
    "Please wait, consulting the Ramones:");
}

main() {
  ramones();
}
```

The program above makes a http get request\cite{3_13}
to a document with some phrases (one per line), choose a random line and print
the phrase on it. If we try to run this program now, we'll have some errors:

```
$ dart _request.dart
Unhandled exception:
Could not resolve a package location for base at file:///home/taiar/dev/monografia/mono1/dart/get_request/get_request.dart
#0      _handlePackagesReply (dart:_builtin:416)
#1      _RawReceivePortImpl._handleMessage (dart:isolate-patch/isolate_patch.dart:148)
```

As the error is telling us, the program _could not resolve the package location_.
That happens because the command ```import 'package:http/http.dart' as http;```
is trying to use a package that is not from Dart's default SDK and is not
present in our environment. Dart SDK ships with a program called **pub**. With
this tool we can define our program's dependencies and install all the possible
package's dependencies these packages need to work. To configure the project's 
dependencies, we'll have to use the pub specifications. We define the program's
dependencies with a file called _pubspec.yaml_ in the root of our project. For 
this example, the pubspec.yaml should look like this:

```yaml
name: request
dependencies:
  http: any
```

It says that, for our project called _request_, we have a dependency with the
package _http_, at _any_ version (we are assuming that all _http_ package
versions have the functionalities we need to run our program). To install the
dependencies, we have to run the command:

```
$ pub get
Resolving dependencies... (5.5s)
+ charcode 1.1.0
+ collection 1.1.3
+ convert 1.0.0
+ crypto 0.9.1
+ http 0.11.3+2
+ http_parser 1.1.0
+ path 1.3.6
+ source_span 1.2.1
+ stack_trace 1.4.2
+ string_scanner 0.1.4
+ typed_data 1.0.0
Changed 11 dependencies!
```

As you can see, it downloads lots of other packages, all they are dependencies
for the _http_ package (and, or course, for our program). All done, if we try to
run the program again, it might work correctly:

```
$ dart get_request.dart
Please wait, consulting the Ramones:
& I won't be back till monday
```

#### Asynchronous Programming

Dart is a single-threaded programming language. If any code blocks the thread of
execution (for example, by waiting for a time-consuming operation like some
heavy I/O operation or a network request) the program effectively freezes. 
Asynchronous operations let your program run without getting blocked. Dart uses 
Future objects to represent asynchronous operations.

Lets make a new version of our previous _request.dart_ program. This new version
will make 2 HTTP Get request: the usual Ramones quotes and the other one teach
us how to say "Merry Christmas" in various languages of the world. This is the
code:

```dart
import 'dart:math';
import 'package:http/http.dart' as http;

readFromRemoteFile(url, sentence) {
  var rng = new Random();
  print(sentence);
  http.get(url).then((val){
    var texto = val.body;
    var frases = texto.split("\n");
    print(frases[rng.nextInt(frases.length)]);
  });
}

ramones() {
  readFromRemoteFile("http://aurelio.net/doc/ramones.txt",
    "Please wait, consulting the Ramones:");
}

natal() {
  readFromRemoteFile("http://www.crossladies.com/feliz_natal.txt",
    "This is how we say 'Merry Christmas' in:");
}

main() {
  ramones();
  natal();
}
```

Running the program, we would have a result like:

```
Please wait, consulting the Ramones:
This is how we say 'Merry Christmas' in:
Havaiano  Mele Kalikimaka
Come along (surfin') baby wait and see (surfin' safari)
```

There is one problem with the program. The program has race conditions. 
When the ```ramones``` function is called, it prints a message on the screen and
make the HTTP get request, setting a callback function that choose the line and 
prints the message when the response of the request returns. Before the response 
returns, the ```natal``` function is called, printing the message and waiting 
for another response from HTTP. Then the two responses are back (on diferent 
times) and printed too.

Dart has the concept of **Future**. A Future represents a means for getting a
value sometime in the future \cite{3_14} 
\cite{3_15} \cite{3_16}.
When a function that returns a Future is invoked, two things happen:

1. The function queues up work to be done and returns an uncompleted Future object.
2. Later, when a value is available, the Future object completes with that value
   (or with an error; we’ll discuss that later).

Lets rewrite the program solving the race conditions with futures to see how it 
works:

```dart
import 'dart:math';
import 'package:http/http.dart' as http;

readFromRemoteFile(url, sentence) async {
  var rng = new Random();
  print(sentence);
  var response = await http.get(url);
  var texto = response.body;
  var frases = texto.split("\n");
  print(frases[rng.nextInt(frases.length)]);
}

ramones() async {
  await readFromRemoteFile("http://aurelio.net/doc/ramones.txt",
    "Please wait, consulting the Ramones:");
}

natal() async {
  await readFromRemoteFile("http://www.crossladies.com/feliz_natal.txt",
    "This is how we say 'Merry Christmas' in:");
}

main() async {
  await ramones();
  natal();
}
```

We have all those asynchronous methods. We mark'em as asynchronous. We tell the 
program, which methods it must _await_ the result for continue the execution. 
That's all!