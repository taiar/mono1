Rust was made to be a system's programming language. There are lots of ways to
integrate Rust programs into computer systems and other programming languages.
In this usage example we'll see how to write a system shared library in Rust and
use it from other programs in other languages.

\subsection{Our problem}

This is the Monte Carlo algorithm for a Pi approximation 
 \cite{8_1} written in Python:

\begin{lstlisting}[label=pytthth,caption=Python Monte Carlo algorithm]
import random
import sys

def montecarlopi(n):
    m = 0.
    n = int(n)
    for i in range(0, n):
        x, y = random.random(), random.random()
        if ((x ** 2) + (y ** 2)) < 1:
            m = m + 1
    return 4 * m / n

print montecarlopi(sys.argv[1])
\end{lstlisting}

It computes a approximation for the Pi value, based in the given parameter. When
bigger is the parameter, better is the approximation value. Let's measure some
executions:

\begin{verbatim}
$ time python pi.py 10
3.6

real    0m0.016s
user    0m0.015s
sys 0m0.000s
\end{verbatim}

The programs is fast and the value is a very bad approximation. Let's increase
the precision:

\begin{verbatim}
$ time python pi.py 100000
3.14184

real    0m0.067s
user    0m0.058s
sys 0m0.008s

$ time python pi.py 1000000
3.140332

real    0m0.533s
user    0m0.504s
sys 0m0.028s

$ time python pi.py 10000000
3.1423776

real    0m5.192s
user    0m4.912s
sys 0m0.276s

$ time python pi.py 100000000
3.14162228

real    3m55.038s
user    0m52.011s
sys 0m6.175s
\end{verbatim}

The program became very slow very fast. It's a common approach to implement the
slow algorithms of the systems in platforms that runs fast and integrate into
the real workflow by calling this solution in other platform. So, let's take a
look on the implementation of this algorithm in Rust:

\begin{lstlisting}[label=rmcalg,caption=Rust Monte Carlo algorithm]
extern crate rand;

use std::env;
use std::str::FromStr;
use rand::distributions::{IndependentSample, Range};

fn montecarlopi(n: u32) -> f32 {
    let between = Range::new(-1f32, 1.);
    let mut rng = rand::thread_rng();

    let total = n;
    let mut in_circle = 0;

    for _ in 0..total {
        let a = between.ind_sample(&mut rng);
        let b = between.ind_sample(&mut rng);
        if a*a + b*b <= 1. {
            in_circle += 1;
        }
    }
    4. * (in_circle as f32) / (total as f32)
}

fn main() {
   let args: Vec<String> = env::args().collect();
   if args.len() > 1 {
     println!("{}", montecarlopi(FromStr::from_str(&args[1]).unwrap()));
   }
}
\end{lstlisting}

It Is the same algorithm at all. Let's measure the execution of this program:

\begin{verbatim}
$ time ./release/point 100000000
3.1413338

real    0m5.323s
user    0m1.911s
sys 0m3.410s

time ./release/point 1000000000
3.1415544

real    0m53.193s
user    0m19.046s
sys 0m34.107s
\end{verbatim}

Awesome! The execution time is much better (even with more precision). Now we
should make it callable from Python and other platforms.

\subsection{A Rust library}

Let's start creating a new project with Cargo:

\begin{verbatim}
$ cargo new montepy
$ cd montepy
\end{verbatim}

Now, edit the ./src/lib.rs file Cargo creates inserting our previous code with
some changes I'll explain later:

\begin{lstlisting}[label=rlllabrar,caption=Rust library code]
extern crate rand;

use std::env;
use std::str::FromStr;
use rand::distributions::{IndependentSample, Range};

#[no_mangle]
pub extern fn montecarlopi(n: u32) -> f32 {
    let between = Range::new(-1f32, 1.);
    let mut rng = rand::thread_rng();

    let total = n;
    let mut in_circle = 0;

    for _ in 0..total {
        let a = between.ind_sample(&mut rng);
        let b = between.ind_sample(&mut rng);
        if a*a + b*b <= 1. {
            in_circle += 1;
        }
    }
    4. * (in_circle as f32) / (total as f32)
}
\end{lstlisting}

The not so odd modification is the $pub extern$ annotations in the font of
the $montecarlopi$ function. The $pub$ means that this function should be
callable from outside of this module, and the $extern$ says that it should be
able to be called from C. Also the $main$ method was removed because we don't
want this program to execute itself. The other modification was the $\#[no_mangle]$ annotation over the method. When you create a Rust library, it
changes the name of the function in the compiled output. This attribute turns
that behavior off  \cite{8_2}.

Also we had to change the $Cargo.toml$ file. It will look like this:

\begin{verbatim}
[package]
name = "montepy"
version = "0.1.0"
authors = ["André Taiar <taiar@dcc.ufmg.br>"]

[dependencies]
rand = "*"

[lib]
name = "montepi"
crate-type = ["dylib"]
\end{verbatim}

The additions here where the $dependencies$ clause, specifying that rand library
is our dependence (Cargo will download and install it for us), the $lib$ clause,
specifying we are building a library and we want to compile our library into a
standard dynamic library  \cite{8_3}. Let's build
the library:

\begin{verbatim}
$ cargo build --release
    Updating registry `https://github.com/rust-lang/crates.io-index`
   Compiling winapi v0.2.5
   Compiling libc v0.2.2
   Compiling winapi-build v0.1.1
   Compiling advapi32-sys v0.1.2
   Compiling rand v0.3.12
   Compiling montepy v0.1.0 (file:///home/taiar/dev/mono1/rust/montepy)
\end{verbatim}

Now in the $./target/release/$ directory there is a file called
$libmontepi.so$ which is our compiled dynamic library. Now, we can create a
python program that uses this library and compute our result very fast. Here is
the code:

\begin{lstlisting}[label=perrr,caption=Python wrong code for calling the library]
from ctypes import cdll

lib = cdll.LoadLibrary("target/release/libmontepi.so")
print lib.montecarlopi(100000000)

print("done!")
\end{lstlisting}

The execution result:

\begin{verbatim}
$ time python pi.py 
22862944
done!

real    0m5.412s
user    0m2.061s
sys 0m3.343s
\end{verbatim}

The program runs really fast but wait, there is a problem. The aproximation
result is totally wrong.

The problem is that the integers and floats we use in our implementation of the
Rust algorithm has different forms from the types of Python. To get around this,
we must convert our numbers for types that are compatible with our system. Here
is our new Python code:

\begin{lstlisting}[label=pytonright,caption=Python right code]
from ctypes import cdll
import ctypes

lib = cdll.LoadLibrary("target/release/libmontepi.so")

montecarlopi = lib.montecarlopi
montecarlopi.restype = ctypes.c_float
print montecarlopi(100000000)

print("done!")
\end{lstlisting}

\begin{verbatim}
$ time python pi.py 
3.14137887955
done!

real  0m5.400s
user  0m1.952s
sys 0m3.447s
\end{verbatim}

and in Node.js (with the ffi package  \cite{8_4}):

\begin{lstlisting}[label=njs,caption=Node calling our Rust library]
var ffi = require('ffi');

var pipi = ffi.Library('./target/release/libmontepi.so', {
    'montecarlopi': ['float', ['int']]
});

console.log(pipi.montecarlopi(100000000));
\end{lstlisting}

\begin{verbatim}
$ time nodejs pi.js 
3.1415293216705322

real  0m5.466s
user  0m2.038s
sys 0m3.420s
\end{verbatim}

and in C (the $libmontepi.so$ file must be available in your system):

\begin{lstlisting}[label=ccorl,caption=C calling our Rust library]
#include <stdint.h>
#include <stdio.h>

float montecarlopi(const int precision);

int main() {
  printf("%f\n", montecarlopi(100000000));
  return 0;
}
\end{lstlisting}

\begin{verbatim}
$ gcc -o pi pi.c -lmontepi
$ time ./pi 
3.141783

real  0m5.367s
user  0m1.925s
sys 0m3.441s
\end{verbatim}
