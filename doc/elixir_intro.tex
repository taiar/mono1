
\subsection{How did it appear?}

Elixir \cite{5_1} is a programming language
designed to build scalable and maintainable systems. It first appeared in may
2012 after 1 year of \textit{underground development} and at first it was a solo
project from its creator and maintainer José
Valim \cite{5_2}. In the begining of 2012,
Plataformatec \cite{5_3}, a brazilian tech company
where José Valim works (and is one of the owners) starts to sponsor and support
the project and then the language got a real improvement.

Even before the first stable version was launched, Elixir attract many
contributors and enthusiasts. It counts 6 written books about programming in
Elixir and other recipes \cite{5_4}, 3
international conferences \cite{5_5} and 300+
open source contributors \cite{5_6}. Besides having
its own community, Elixir is also attracts members of the Ruby and Erlang
communities too.

The version 1.0 of the language \cite{5_7} was
released in September 10 in the year of 2014.

\subsection{Why was it designed and implemented?}

Back in 2010, José Valim was a very active member of the Ruby
Language \cite{5_8} community. He worked on the
core team of a major Ruby open source project called Ruby on
Rails \cite{5_9} (a framework to build web
applications). One of his contributions for this tool was to make the Ruby on
Rails framework **thread safe**. At first, Ruby on Rails applications doesn't
scaled horizontally. At that time, there was no advantage on running a Rails
application on multi-core machines. Making the framework thread safe means that
the application written in Ruby on Rails when deployed in multi-core machines
would run efficiently and use all processing power of that environment and in
the right way (there will exist no problems with concurrency). Valim finds out
that, on that time, there weren't good tools on Ruby ecosystem to work with
concurrency and he started do research about this topics trying to find in other
languages and platforms the best ways to solve the problem they had in Rails.

During his explorations, Valim read the awesome \textbf{7 languages in 7 weeks} 
\cite{5_10} where he meets the Erlang \cite{5_11} programming language. Erlang
is a functional programming language and it has a virtual machine. It was
developed by Ericsson \cite{5_12} and open sourced in 1998. It was designed for
distributed and fault tolerant applications that runs in real time in a
uninterrupted environment.

Very interested in the platform, he started to look deeper in the Erlang
ecossystem writing programs and projects in Erlang. He loved the things the
language has but misses things that he thought to be very important like good
metaprogramming and polimorphysm. Despite these misses, the Erlang Virtual
Machine was really awesome and Erlang was very good in concurrency.

At that time, there was only one language that had all the things Valim wishes
to have in a language and it was Clojure \cite{5_13}.
Clojure is a dynamic language (no statyc typing), very extensible and with great
focus on concurrency. Ruby was a dynamic and extensible language but was not
good in concurrency scenarios like we saw. Go \cite{5_14}
has a focus in concurrency but is static typed and not a very extensible
language. Clojure has all he wants but is a JVM language (language that runs in
the Java Virtual Machine) and he thinks that was a good idea to have other
language option on this niche.

\subsection{How can we use it, e.g., install and run?}

The only prerequisite to run Elixir programs is Erlang 17.0 or later. So, before
install the Elixir you must be sure to have Erlang installed.

\subsubsection{Linux}

There are easy install setups for all major Linux distributions. Most of them
installs the Erlang Virtual Machine too, so yout don't have do care with it. You
can install on Ubuntu based distros with the commands:

\begin{verbatim}
$ wget https://packages.erlang-solutions.com/erlang-solutions_1.0_all.deb && sudo dpkg -i erlang-solutions_1.0_all.deb
$ sudo apt-get update
$ sudo apt-get install elixir
\end{verbatim}

on Red Hat distros with:

\begin{verbatim}
$ yum install elixir
\end{verbatim}

and on Arch Linux with the command:

\begin{verbatim}
$ pacman -S elixir
\end{verbatim}

\subsubsection{Mac OS}

On Mac OS you can install using Homebrew:

\begin{verbatim}
$ brew update
$ brew install elixir
\end{verbatim}

\subsubsection{Windows}

There are precompilled packages for Windows you can download and install on the
Elixir download session at the website. You can also use the Chocolatey tool
and install with:

\begin{verbatim}
cinst elixir
\end{verbatim}

\subsubsection{Testing the installation}

After installing, let's check if we installed right and we can run Elixir
programs in our environment. Elixir code can be scripted and interpreted or
compiled into bytecode that runs on the Erlang VM. Let's checkt the interpreter.

Write and save the following code to the $hello.exs$ file:

\begin{lstlisting}[label=ehw,caption=Elixir Hello World]
IO.puts "Hello, Elixir!"
\end{lstlisting}

and then, run the commands:

\begin{verbatim}
$ elixir hello.exs
\end{verbatim}

and you'll have the $Hello, Elixir!$ sentence on your screen.

For the compilation to bytecodes, the Elixir program must be inside a module.
Write the $hello.ex$ file with the following contents:

\begin{lstlisting}[label=emhw,caption=Elixir Hello World inside a module]
defmodule Hello do
  def say do
    IO.puts "Hello, Elixir!"
  end
end
\end{lstlisting}

and compile with the command:

\begin{verbatim}
$ elixirc hello.ex
\end{verbatim}

Check that the $Elixir.Hello.beam$ file is created with bytecode in the same
directory. Let's check if the module works using the Elixir interactive
environment. Run the program in the command line in this same directory:

\begin{verbatim}
$ iex
Erlang/OTP 18 [erts-7.1] [source] [smp:4:4] [async-threads:10] [kernel-poll:false]

Interactive Elixir (1.1.1) - press Ctrl+C to exit (type h() ENTER for help)
iex(1)>
\end{verbatim}

In the interactive execution environment, type the code:

\begin{verbatim}
iex(1)> Hello.say()
Hello, Elixir!
:ok
\end{verbatim}

and check that it loads the module and executes the function we defined.

\subsection{Simple programs}

Elixir is a functional language (the only functional-only language I learned on
this work). Here are its basic types:

\begin{verbatim}
iex> 1          # integer
iex> 0x1F       # integer
iex> 1.0        # float
iex> true       # boolean
iex> :atom      # atom / symbol
iex> "elixir"   # string
iex> [1, 2, 3]  # list
iex> {1, 2, 3}  # tuple
\end{verbatim}

There are some common features in functional languages that we'll
look on the next examples.

\subsubsection{Anonymous functions}

In the interactive execution environment, let's see some things about anonymous
functions:

\begin{verbatim}
$ iex
Erlang/OTP 18 [erts-7.1] [source] [smp:4:4] [async-threads:10] [kernel-poll:false]

Interactive Elixir (1.1.1) - press Ctrl+C to exit (type h() ENTER for help)
iex(1)> add = fn a, b -> a + b end
#Function<12.54118792/2 in :erl_eval.expr/5>
iex(2)> is_function(add)
true
iex(3)> add
#Function<12.54118792/2 in :erl_eval.expr/5>
iex(4)> add.(2, 4)
6
iex(5)> add_ten = fn a -> add.(a, 10) end
#Function<6.54118792/1 in :erl_eval.expr/5>
iex(6)> add_ten.(11)
21
iex(7)> x = add_ten.(0)
10
iex(8)> x
10
iex(9)> (fn -> x = 42 end).()
42
iex(10)> x
10
\end{verbatim}

Functions are delimited by the keywords **fn** and **end**. They can be assigned
to variables and passed as function parameters just as other types in the
language. In the example, we assing a function that add two variables to a
variable called $add$. To test it, we invoke the anonymous function with $add.$
(we need that point after the variable name to call the anonymous functions) and
then the parameters.

Anonymous functions are clojures and they can access variables that are in scope
when the function is defined, like in the example when we create the $add_ten$
variable. The variable assigned inside a function doesn't affects the outer
scope, so we check it on the last part of the execution.

\subsubsection{Lists, tuples and immutability}

Here are some operations with lists:

\begin{verbatim}
$ iex
Erlang/OTP 18 [erts-7.1] [source] [smp:4:4] [async-threads:10] [kernel-poll:false]

Interactive Elixir (1.1.1) - press Ctrl+C to exit (type h() ENTER for help)
iex(1)> music = [:a, :b, :c] ++ [1, 2, 3]
[:a, :b, :c, 1, 2, 3]
iex(2)> music -- [:c, 3]
[:a, :b, 1, 2]
iex(3)> hd(music)
:a
iex(4)> tl(music)
[:b, :c, 1, 2, 3]
\end{verbatim}

There are the concatenation operator **++** and the subtractor **--** operators
that concatenates and subtract lists and there are the **hd** and **tl**
functions that returns the head and the tail of the list.

\begin{verbatim}
$ iex
Erlang/OTP 18 [erts-7.1] [source] [smp:4:4] [async-threads:10] [kernel-poll:false]

Interactive Elixir (1.1.1) - press Ctrl+C to exit (type h() ENTER for help)
iex(1)> tuple = {:ok, "hello"}
{:ok, "hello"}
iex(2)> put_elem(tuple, 1, "world")
{:ok, "world"}
iex(3)> elem(tuple, 0)
:ok
iex(4)> tuple_size(tuple)
2
iex(5)> elem(tuple, 1)
"hello"
\end{verbatim}

We create a tuple using curly brackets. We can set a value for a determined
tuple position, return a position and get a tuple size. Note in the last command
that the second element of the variable tuple remains the same. The original
tuple stored in the tuple variable was not modified because Elixir data types
are immutable. By being immutable, Elixir code is easier to reason about as you
never need to worry if a particular code is mutating your data structure in
place.

Lists in Elixir are linked. This means that each element in a list holds its
value and points to the following element until the end of the list is reached.
Accessing the length of a list is a linear operation: we need to traverse the
whole list in order to figure out its size. Updating a list is fast as long as
we are prepending elements.

Tuples are stored contiguously in memory. This means getting the tuple size or
accessing an element by index is fast. However, updating or adding elements to
tuples is expensive because it requires copying the whole tuple in memory. Those
performance characteristics dictate the usage of those data structures.

\subsubsection{Pattern matching}

Pattern matching allows to easily destructure data types such as tuples and
lists. It is one of the foundations of recursion in Elixir. Here are some
examples matching lists and tuples.

\begin{verbatim}
$ iex
Erlang/OTP 18 [erts-7.1] [source] [smp:4:4] [async-threads:10] [kernel-poll:false]

Interactive Elixir (1.1.1) - press Ctrl+C to exit (type h() ENTER for help)
iex(1)> {a, b, c} = {:hello, "world", 42}
{:hello, "world", 42}
iex(2)> b
"world"
iex(3)> [a, b, c] = [1, 2, 3]
[1, 2, 3]
iex(4)> b
2
iex(5)> list = [1, 2, 3]
[1, 2, 3]
iex(6)> [0|list]
[0, 1, 2, 3]
iex(7)> [head | tail] = [1, 2, 3]
[1, 2, 3]
iex(8)> head
1
iex(9)> tail
[2, 3]
\end{verbatim}

\subsubsection{Modules}

Elixir groups several functions into modules. In order to create our own modules
we use the **defmodule** macro. We use the **def** macro to define functions in
that module.

\begin{lstlisting}[label=esum,caption=Elixir Module example]
defmodule Math do
  def sum(a, b) do
    a + b
  end
end

Math.sum(1, 2)
\end{lstlisting}

Inside a module, we can define functions with **def** and private functions with
**defp**. A function defined with **def** can be invoked from other modules
while a private function can only be invoked locally.

\begin{lstlisting}[label=econc,caption=Elixir pattern matching and function visibility]
defmodule Concato do

  def concat(a, b) do
    do_concat(a, b)
  end

  defp do_concat([h|t], [a|b]) do
    [h|t] ++ [a|b]
  end

  defp do_concat(a, b) do
    a <> b
  end

end

IO.puts Concato.concat("Merry", "Christmas") #=> MerryChristmas
IO.inspect Concato.concat([:a, :b, :c], [1, 2, 3]) #=> [:a, :b, :c, 1, 2, 3]
\end{lstlisting}

The last example show the usage of a module with two private functions and a
public function. The public function is used for interface with the module and
it calls the possible 2 other functions using pattern matching. When the
arguments are two lists, it calls a function that concatenate lists and when
they are strings, the function that concatenates strings is called.

\subsubsection{Recursion}

Due to immutability, loops in Elixir (as in any functional programming language)
are written differently from imperative languages. Functional languages rely on
recursion: a function is called recursively until a condition is reached that
stops the recursive action from continuing.

\begin{lstlisting}[label=errrr,caption=Elixir recursion]
defmodule Recursion do

  def print_multiple_times(msg, n) when n <= 1 do
    IO.puts msg
  end

  def print_multiple_times(msg, n) do
    IO.puts msg
    print_multiple_times(msg, n - 1)
  end

end

Recursion.print_multiple_times("Gol!", 7)
\end{lstlisting}

Let's use recursion to make some math operations. The example above has three
**public** functions: the first one return the sum of all numbers in the list
and the second return the list with all elements multiplied by two.

\begin{lstlisting}[label=elists,caption=Elixir recursion with lists]
defmodule Math do

  def sum_list([h|t]) do
    sum_list([h|t], 0)
  end

  defp sum_list([head|tail], accumulator) do
    sum_list(tail, head + accumulator)
  end

  defp sum_list([], accumulator) do
    accumulator
  end

  def double_each([head|tail]) do
    [head * 2|double_each(tail)]
  end

  def double_each([]) do
    []
  end

end

IO.puts Math.sum_list([2,3,5]) #=> 10
IO.inspect Math.double_each([2,3,5]) #=> [4, 6, 10]
\end{lstlisting}

Elixir has a Enum \cite{5_15} module that
provides a set of algorithms that enumerate over collections. This module has a
**reduce** and a **map** functions that we'll use to rewrite our last example:

\begin{lstlisting}[label=eenum,caption=Elixir lists with Enum]
defmodule Mathenum do

  def sum_list([h|t]) do
    Enum.reduce([h|t], 0, fn(x, acc) -> x + acc end)
  end

  def double_each([h|t]) do
    Enum.map([h|t], fn(x) -> x * 2 end)
  end

end

IO.puts Mathenum.sum_list([2,3,5]) #=> 10
IO.inspect Mathenum.double_each([2,3,5]) #=> [4, 6, 10]
\end{lstlisting}