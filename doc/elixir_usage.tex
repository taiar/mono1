On this usage example, we'll use most of the concepts on the last article to 
build a more complex program that explores the concurrent features of Elixir.
Starting with a Fibonacci function we'll increment the program touching concepts
that were not in the first article like pipes, processes, communication between 
processes and tail call optimization.

\subsection{Fibonacci}

Let's begin with a canonical starter recursive Fibonacci implementation with 
Elixir. Here is the code:

\begin{lstlisting}[label=efib1,caption=Classic Fibonacci]
defmodule Fib do

  def fib(0) do 0 end
  def fib(1) do 1 end
  def fib(n) do fib(n - 1) + fib(n - 2) end

end

IO.puts Fib.fib(0)
IO.puts Fib.fib(1)
IO.puts Fib.fib(2)
IO.puts Fib.fib(3)
IO.puts Fib.fib(4)
IO.puts Fib.fib(5)
IO.puts Fib.fib(6)
IO.puts Fib.fib(7)
IO.puts Fib.fib(8)
\end{lstlisting}

Nothing new here, just right to the point. Just write the problem's definition 
and we got the right result:

\begin{verbatim}
  
\end{verbatim}

\begin{verbatim}
$ elixir fib.ex 
0
1
1
2
3
5
8
13
21
\end{verbatim}

\subsection{Many Fibonacci numbers}

Let's improve our last code to generate many numbers passing them as a list to 
the function \cite{6_2}. Here is the code:

\begin{lstlisting}[label=emf,caption=Many Fibonacci numbers]
defmodule Fib do

  def run(list) do
    list
    |> Enum.map(&(fib(&1)))
    |> inspect
    |> IO.puts
  end

  def fib(0) do 0 end
  def fib(1) do 1 end
  def fib(n) do fib(n - 1) + fib(n - 2) end

end

Fib.run([0,1,2,3,4,5,6,7,8])
\end{lstlisting}

The $|>$ symbol used in the snippet above is the pipe operator: it simply 
takes the output from the expression on its left side and passes it as the first
argument to the function call on its right side. It’s similar to the Unix **|** 
operator. Its purpose is to highlight the flow of data being transformed by a 
series of functions. The $run$ function without the pipe operator would be:

\begin{lstlisting}[label=dartMap,caption=Dart dynamic type]
def run(list) do
  IO.puts(inspect(Enum.map(list, fn(x) -> fib(x) end )))
end
\end{lstlisting}

The $Enum.map$ function calls the function on the second parameter for each 
element in the list passed as the first parameter returning a new list with the 
transformed elements. The $inspect$ function returns a String with the list
\"pretty printed\" (written in a human readable format) and we print the list with
the $IO.puts$ function. The result of the execution, for whatever $run$ 
implementations is:

\begin{verbatim}
$ elixir fib.ex 
[0, 1, 1, 2, 3, 5, 8, 13, 21]
\end{verbatim}


\subsection{Parallel Fibonacci numbers}

In our last example, we run the function that calculates each Fibonacci's number
sequentially. The execution waits each of the numbers to be calculated before 
calculate the next number. Running this program for some bigger numbers would be
slow. Let's measure the time to calculate the Fibonacci numbers from 30 to 40:

\begin{verbatim}
$ time elixir fib.ex 
[832040, 1346269, 2178309, 3524578, 5702887, 9227465, 14930352, 24157817, 39088169, 63245986, 102334155]

real  1m2.186s
user  1m1.664s
sys 0m0.684s
\end{verbatim}

Now I'll rewrite the program to calculate these numbers using data parallelism.
Our code will distribute the number calculation across different parallel 
computer cores possibly making the execution faster. Here is the code:

\begin{lstlisting}[label=epf,caption=Parallel processing for Fibonacci in Elixir]
defmodule Fib do

  def run(list) do
    list
    |> Enum.with_index
    |> Enum.map fn(ni) -> spawn_run(self, ni) end
    receive_fibs(length(list), [])
  end

  defp receive_fibs(lns, result) do
    receive do
      fib ->
        result = [fib | result]
        if lns == 1 do
          IO.puts(print_fibs(result))
        else
          receive_fibs(lns - 1, result)
        end
    end
  end

  defp print_fibs(fibs) do
    fibs
    |> Enum.sort(fn({_, a}, {_, b}) -> a < b end)
    |> Enum.map(fn({f, _}) -> f end)
    |> inspect
  end

  def spawn_run(pid, ni) do
    spawn __MODULE__, :send_run, [pid, ni]
  end

  def send_run(pid, {n, i}) do
    send pid, {fib(n), i}
  end

  def fib(0) do 0 end
  def fib(1) do 1 end
  def fib(n) do fib(n - 1) + fib(n - 2) end
end

Fib.run([30,31,32,33,34,35,36,37,38,39,40])
\end{lstlisting}

Our already existent $run$ function was modified. The $Enum.with_index$ 
method returns the collection with each element wrapped in a tuple alongside its
index. This way we can identify this number later to return them in the correct
order. The $Enum.map$ function that used to call the $fib$ function now 
calls a $spawn_run$ function. At the end of the $run$ function it calls a 
$receive_fibs$ function.

The $spawn_run$ function just spawn off new processes \cite{6_1}
of execution. The Elixir's $spawn$ function takes a function which it will 
execute in another process. It returns a PID (process identifier). The spawned 
process will execute the given function and exit after the function is done. The
program uses the $send$ and $receive$ functions to comunicate between 
different proccesses. The $send$ method receives a PID and a message and sends
the message to the proccess with the referred PID. The $receive$ function has 
one parameter that must be a function that will receive the potentially message 
sent. 

So in our code, the $spawn_run$ is called (for each one of the numbers we want
the fibonacci number) and it calls a $send_run$ function. The $send_run$ 
function sends back a result of our already known $fib$ functions. The receive
function is used inside the $receive_fibs$ function. $receive_fibs$ works as
a synchronization function that awaits the execution of all parallel $fib$ 
calls, adding them to a list and returning this list when there is no more 
proccesses running.

When all the execution is over, the $receive$ method calls the $print_fibs$
function. This function orders the calculated number by its index (mentioned on 
the $run$ function). This is necessary because the result of the parallel 
and asynchronous execution times are not deterministic and the order it returns 
the numbers is unknown. At last, the tuples with indexes are removed and just 
the result of the Fibonacci numbers are printed on the screen.

Now, this is the execution time of the parallel version of the program:

\begin{verbatim}
$ time elixir fib.ex 
[832040, 1346269, 2178309, 3524578, 5702887, 9227465, 14930352, 24157817, 39088169, 63245986, 102334155]

real  0m33.141s
user  1m22.564s
sys 0m1.208s
\end{verbatim}

The real execution time of the parallel program is almost the half of the 
execution time for the sequential version of the code. This time would become 
even better if the machine that executes the program has a processor with a 
bigger number of execution cores.

\subsection{Tail call optimization}

Tail-call optimization is where you are able to avoid allocating a new stack 
frame for a function because the calling function will simply return the value 
that it gets from the called function.

The parallel portion of the implementations just remains the same. The 
modification is on the $fib$ functions that now has a tail call, avoiding so 
many recursion calls. Here is the code:

\begin{lstlisting}[label=dartMap,caption=Parallel Fibonacci with tail call optimization]
defmodule Fib do

  def run(list) do
    list
    |> Enum.with_index
    |> Enum.map fn(ni) -> spawn_run(self, ni) end
    receive_fibs(length(list), [])
  end

  defp receive_fibs(lns, result) do
    receive do
      fib ->
        result = [fib | result]
        if lns == 1 do
          IO.puts(print_fibs(result))
        else
          receive_fibs(lns - 1, result)
        end
    end
  end

  defp print_fibs(fibs) do
    fibs
    |> Enum.sort(fn({_, a}, {_, b}) -> a < b end)
    |> Enum.map(fn({f, _}) -> f end)
    |> inspect
  end

  def spawn_run(pid, ni) do
    spawn __MODULE__, :send_run, [pid, ni]
  end

  def send_run(pid, {n, i}) do
    send pid, {fib(n), i}
  end

  def fib(n) do fib(n, 1, 0) end
  def fib(0, _, _) do 0 end
  def fib(1, a, b) do a + b end
  def fib(n, a, b) do fib(n - 1, b, a + b) end

end

Fib.run([30,31,32,33,34,35,36,37,38,39,40])
\end{lstlisting}

It might be more illustrative if shown in a non-functional language, like Python:

\begin{lstlisting}[label=ppppp,caption=Python tail call optimization for Fibonacci]
def fib(i, current = 0, next = 1):
  if i == 0:
    return current
  else:
    return fib(i - 1, next, current + next)
\end{lstlisting}

We can see that now, the real execution time of our program is very smaller than
our two last versions:

\begin{verbatim}
$ time elixir fib.ex 
[832040, 1346269, 2178309, 3524578, 5702887, 9227465, 14930352, 24157817, 39088169, 63245986, 102334155]

real  0m2.321s
user  0m2.152s
sys 0m0.320s
\end{verbatim}
